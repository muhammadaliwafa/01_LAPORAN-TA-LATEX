% !TeX spellcheck = en_US
%\renewcommand{\thefootnote}{\arabic{footnote}}
\chapter{PENDAHULUAN}
\label{BAB1:pendahuluan}

\section{Latar Belakang}
Transformator daya merupakan salah satu peralatan dalam sistem kelistrikan yang memiliki peran fundamental. Dalam pengoperasiannya transformator berperan dalam menaikkan serta menurunkan tegangan pada jaringan transmisi. Apabila transformator daya tidak dapat bekerja dengan baik maka dapat menurunkan kualitas listrik atau lebih lanjut dapat menyebabkan terhentinya pelayanan listrik yang diterima oleh konsumen. Selain itu apabila transformator daya dalam kondisi tidak normal dapat membuat rugi-rugi daya menjadi semakin besar, hal ini tentunya dapat merugikan penyedia listrik. Menurut manufaktur usia transformator daya diperkirakan antara 25-40 tahun, tetapi terkadang terdapat transformator yang usianya di bawah range usia minimal telah rusak \cite{jahromi2009approach}.\par
Pemeliharaan transformator daya sangat penting dilakukan untuk memastikan agar selalu dapat beroperasi dengan baik. Namun jika pemeliharaan dilakukan dengan intensitas yang tinggi tentunya dapat membuat dana yang harus dialokasikan semakin besar. Sedangkan diketahui bahwa transformator merupakan komponen yang membutuhkan hampir 60\% dari biaya total pada gardu induk \cite{jahromi2009approach}. Sehingga diperlukan penjadwalan agar proses pemeliharaan dapat dilakukan secara efektif. Pada dasarnya kondisi sebuah transformator daya dapat diketahui berdasarkan beberapa metode seperti DGA (Dissolve Gas Analysis), pengujian minyak trafo, serta furan.\par
Metode DGA memungkinkan bagi teknisi operator dalam mengetahui adanya kontaminan pada minyak transformator daya. Kadar gas kontaminan dapat menjadi indikator kondisi sebuah transformator daya untuk dapat beroperasi secara normal atau tidak. Pada sisi yang lain adanya pengujian pada transformator daya baik pengujian fisik, pengujian elektrik dan pengujian kimia dapat memberikan data penting mengenai kondisi transformator daya. Pada pengujian fisik akan diperoleh kekuatan minyak transformator dalam menahan tekanan fisik. Pada pengujian dapat diperoleh informasi mengenai breakdown voltage yang untuk mengetahui tegangan yang dapat diizinkan beroperasi pada transformator daya. Pengujian kimia berkontribusi dalam memberikan informasi mengenai tingkat keasaman serta kandungan air dalam minyak transformator yang dapat memicu adanya elektron bebas sebagai penghantar listrik dalam isolator. Selain itu adanya furan yang merupakan salah satu kontaminan dalam minyak transformator daya dapat memberikan informasi mengenai estimasi umur kertas isolasi.\par
Data hasil pengujian transformator daya yang dikumpulkan pada waktu yang lama akan memberikan karakteristik tertentu. Hal ini tentunya memberikan informasi solutif dalam mengidentifikasikan kondisi transformator daya. Banyaknya data yang terkumpul akan sangat sulit dalam melakukan analisis dengan cara konvensional. Adanya metode komputasi modern memberikan dapat mempermudah dalam melakukan perhitungan dalam jumlah yang besar. Hal ini melahirkan suatu metode baru dalam mengenali karakteristik dari suatu data melalui perhitungan yang matematis. Metode pengenalan karakteristik data saat ini dikenal dengan Machine Learning, istilah yang diumpamakan komputer dalam mempelajari suatu data.\par
 Seiring dengan berjalannya waktu Machine Learning telah mengalami banyak perubahan yang menciptakan istilah baru yakni Deep Learning. Pada algoritma tersebut dibentuk dengan menirukan jaringan saraf pada manusia sehingga dapat mempelajari suatu data dengan baik. Perkembangan Deep Learning sendiri sudah banyak dilakukan yang menyesuaikan dengan data yang akan diproses, salah satunya adalah \textit{Long Short Term Memory} (LSTM). Algoritma ini dibentuk untuk menangani data yang bersifat sekuensial seperti halnya data yang selalu berubah seiring dengan berjalannya waktu. Keberadaan LSTM dapat menjadi solusi dalam memproses data hasil pengujian pada transformator daya sehingga hasilnya dapat memberikan klasifikasi mengenai indeks kesehatan transformator daya. Hasil klasifikasi indeks kesehatan selanjutnya dapat dimanfaatkan dalam pemeliharaan transformator daya sehingga dapat dilakukan lebih efektif.


\section{Rumusan Masalah}
Merujuk pada latar belakang yang telah disampaikan, maka diperoleh rumusan masalah sebagai berikut:
\begin{enumerate}
	\item Bagaimana perancangan sistem LSTM sehingga dapat mengklasifikasikan indeks kesehatan transformator daya dengan akurasi yang tinggi dan dengan waktu yang lebih cepat?
	\item Bagaimana pengaruh hyperparameter dalam meningkatkan performa dari sistem yang dirancang?
\end{enumerate}

%\section{Hipotesis}
%\lipsum[4]

\section{Batasan Masalah}
Agar perancangan yang diharapkan sesuai dan dapat tercapai, maka dalam perancangan sistem tersebut, ditentukan ruang lingkup perancangan sebagai berikut:
\begin{enumerate}
	\item Perancangan model metode Long Short Term Memory (LSTM) dilakukan pada lingkup klasifikasi pada data pengujian transformator daya
	\item Hasil perancangan dapat diimplementasikan untuk diagnosis indeks kesehatan transformator daya dengan parameter input yang dibutuhkan sesuai dengan dataset yang digunakan
\end{enumerate}

\section{Tujuan Penelitian}
Adapun tujuan perancangan ini terdiri dari:
\begin{enumerate}
	\item Merancang sistem diagnosa indeks kesehatan pada transformator daya menggunakan metode LSTM dengan akurasi di atas 80\%.
	\item Dapat menentukan pengaruh hyperparameter dari metode LSTM sehingga diperoleh sistem yang diharapkan.
\end{enumerate}


\section{Manfaat Perancangan}
\begin{enumerate}
	\item Dapat melakukan pemeliharaan (maintenance) dini pada transformator daya untuk mencegah terjadinya gangguan pada transformator daya.
	\item Dapat menjaga performa dari transformator daya agar dapat bekerja secara normal.
	\item Dapat meminimalisir biaya pemeliharaan karena dapat terjadwal dengan optimal.
\end{enumerate}

%\section{Waktu Pelaksanaan Penelitian}
%\lipsum[8]

