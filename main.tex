%%%%%%%%%%%%%%%%%%%%%%%%%%%%%%%%%%%%%%%%%%%%%%%%%%%%%%%%%%%%%%%
%
%     filename  = "main.tex",
%     version   = "1.1.0",
%     authors   = "Ade Irawan",
%     email     = "adeirawan@universitaspertamina.ac.id"
%
%%%%%%%%%%%%%%%%%%%%%%%%%%%%%%%%%%%%%%%%%%%%%%%%%%%%%%%%%%%%%%%
%
% Gunakan skripsistyle.cls dan compile dengan pdflatex.
%
%%%%%%%%%%%%%%%%%%%%%%%%%%%%%%%%%%%%%%%%%%%%%%%%%%%%%%%%%%%%%%%
%
%  2020/01/06   Created              	Ade
%  2021/02/03   tambah lembar pemisah	Herdiansyah
%  2021/02/22   edit lembar pengesahan  Ade
%
%%%%%%%%%%%%%%%%%%%%%%%%%%%%%%%%%%%%%%%%%%%%%%%%%%%%%%%%%%%%%%%

\documentclass{skripsistyle}
%\usepackage{showframe} % for showing the frame
\usepackage{lipsum} % for generating dummy text

\usepackage{amsmath}
\usepackage{amssymb}
\usepackage{upgreek}
\usepackage{times}
\usepackage{graphicx}
\usepackage{fancyhdr}
\usepackage{sectsty}
\usepackage{indentfirst}
\usepackage{longtable}
\usepackage{tabularx}
\usepackage{hyperref}
\usepackage{csvsimple} % baru
\usepackage[utf8]{inputenc}% baru
\usepackage{pgfgantt} % baru
\usepackage{pgf}
\usepackage{tikz}
\usepackage{tikzscale}
\usepackage{pgfplots}

%\usepackage{rotating}
\usepackage{makecell}
\usepackage{graphicx}
\usepackage{multirow}
\usepackage{colortbl}
%\usepackage{ieeetran}
\usepackage{caption}
\usepackage[numbers]{natbib} % Bibliography with APA Style
\usepackage{geometry}
\usepackage{pdfpages}
 \geometry{
 a4paper,
 left=3cm,
 top=2.5cm,
 bottom=2.5cm,
 right=2.5cm,
 }
 
\setlength{\footskip}{1.3cm}
%% Line Spacing
\linespread{1.103} % Jangan dihapus. Ini untuk memastikan line spacing ~15pt. Cara pengecekan: ketik "\the\baselineskip" di dalam dokumen
%% Paragraph spacing
\setlength{\parskip}{9pt}
%% Title
\usepackage{titlesec}
\usepackage{apptools}
\titleformat{\chapter}[display]{\fontsize{14}{14} \center\bfseries}{\IfAppendix{LAMPIRAN}{BAB} \thechapter}{0pt}{}{}
\titleformat{\section}
  {\fontsize{12}{12}\bfseries}{\thesection}{1em}{}
  
\titlespacing*{\chapter}{0pt}{0pt}{1em}
\titlespacing*{\section}{0pt}{0pt}{0pt}

%% Page number
\pagestyle{fancy}
\fancyhf{}
\rfoot{Universitas Pertamina \hspace{1pt}-\hspace{1pt} \thepage}

% Redefine the plain page style
\fancypagestyle{plain}{%
  \fancyhf{}%
\rfoot{Universitas Pertamina \hspace{1pt}-\hspace{1pt} \thepage}%
}

\renewcommand{\headrulewidth}{0pt}
\DeclareMathOperator*{\argmin}{arg\,min}

%------------------------------------------------------------
% Semua informasi penting yang diperlukan
%------------------------------------------------------------
\title{Diagnosis Indeks Kesehatan Transformator Daya Menggunakan  \textit{Long Short Term Memory} (LSTM)}{}% {Judul} dan {sub judul jika ada}. Jika tidak ada sub judul biarkan {} kosong 
\titleeng{Title}{}% {Judul} dan {sub judul jika ada} dalam bahasa inggris
\author{Muhammad Ali Wafa}{102117002} % Nama Penulis dan NIM
\fakultas{Teknologi Industri}
\prodi{Teknik Elektro}
\tgllulus{15} {Juni 2021} {2021}%{Januari 2020} % Tanggal dan bulan+tahun dipisah
\tglpengesahan{29 Januari 2020} % Tanggal, bulan, dan tahun digabung
\tglpernyataan{29 Januari 2020} % Tanggal, bulan, dan tahun digabung
\pembimbingsatu{Dr.Eng. Muhammad Abdillah, S.T., M.T.}{116153} % Nama dan NIP
\pembimbingdua{Teguh Aryo Nugroho, S.T., M.T.}{116054} % Nama dan NIP
%\pengujisatu{Randi Farmana Putra, M.Si.}{119030} % Nama dan NIP
%\pengujidua{Herminarto Nugroho, M.Sc.}{116056} % Nama dan NIP
\kaprodi{Dr.Eng. Wahyu Kunto Wibowo, S.T., M.Eng.}{116059} % Nama dan NIP
%------------------------------------------------------------

%------------------------------------------------------------
% Comment/Uncomment command di bawah ini untuk menghilangkan/
%                                      menambahkan watermark
%------------------------------------------------------------
%\usepackage{background} % for adding watermark
%\backgroundsetup{anchor = left, hshift = -4.4cm, scale = 1, angle = 0, opacity = 1,
%         contents = {\includegraphics[width = \paperwidth,
%         height = 0.9\paperheight, keepaspectratio] {watermark.jpeg}}}

%% Pemecahan suku kata
\hyphenation{me-ning-kat-kan di-tu-lis-kan me-ne-ri-ma me-la-lu-i Na-mun or-ga-ni-sa-i pe-ner-je-mah pe-me-rin-tah me-nge-na-li aug-men-ta-tion meng-gu-na-kan di-gu-na-kan learn-ing trans-fer-ing bi-sin-do me-nam-pil-kan ber-da-sar-kan me-la-ku-kan}

\begin{document}

\frontmatter
\upcoverpage
\includepdf{Includes/Pemisah.pdf}
\lembarpengesahan
\lembarpernyataan
\abstrakind{Front/abstrakind}
\abstrakeng{Front/abstrakeng}
\katapengantar{Front/katapengantar}
\daftarisi
\daftartabel
\daftargambar

\mainmatter
\includepdf{Includes/Pemisah.pdf}
% Chapters
% !TeX spellcheck = en_US
%\renewcommand{\thefootnote}{\arabic{footnote}}
\chapter{PENDAHULUAN}
\label{BAB1:pendahuluan}

\section{Latar Belakang}
Transformator daya merupakan salah satu peralatan dalam sistem kelistrikan yang memiliki peran fundamental. Dalam pengoperasiannya transformator berperan dalam menaikkan serta menurunkan tegangan pada jaringan transmisi. Apabila transformator daya tidak dapat bekerja dengan baik maka dapat menurunkan kualitas listrik atau lebih lanjut dapat menyebabkan terhentinya pelayanan listrik yang diterima oleh konsumen. Kerugian lainnya dapat membuat rugi-rugi daya menjadi semakin besar, hal ini tentunya dapat merugikan penyedia listrik. Menurut manufaktur, usia transformator daya diperkirakan antara 25-40 tahun, tetapi terkadang terdapat transformator yang usianya di bawah range usia minimal telah rusak \cite{jahromi2009approach}.\par

Pemeliharaan transformator daya sangat penting dilakukan untuk memastikan agar selalu dapat beroperasi dengan baik. Namun, jika pemeliharaan dilakukan dengan intensitas yang tinggi tentunya dapat membuat dana yang harus dialokasikan semakin besar. Sedangkan diketahui bahwa transformator merupakan komponen yang membutuhkan hampir 60\% dari biaya total pada gardu induk \cite{jahromi2009approach}. Sehingga diperlukan penjadwalan agar proses pemeliharaan dapat dilakukan secara efektif. Pada dasarnya kondisi sebuah transformator daya dapat diketahui berdasarkan beberapa metode seperti DGA (Dissolve Gas Analysis), pengujian minyak trafo, serta furan.\par

Metode DGA memungkinkan bagi teknisi operator dalam mengetahui adanya kontaminan pada minyak transformator daya. Kadar gas kontaminan dapat menjadi indikator kondisi sebuah transformator daya untuk dapat beroperasi secara normal atau tidak \cite{duval1989dissolved}. Pada sisi yang lain adanya pengujian pada transformator daya baik pengujian fisik, pengujian elektrik dan pengujian kimia dapat memberikan data penting mengenai kondisi transformator daya. Pada pengujian fisik akan diperoleh kekuatan minyak transformator dalam menahan tekanan fisik \cite{7076831}. Pada pengujian elektrik dapat diperoleh informasi mengenai \textit{breakdown voltage} yang untuk mengetahui tegangan yang dapat diizinkan beroperasi pada transformator daya. Pengujian kimia berkontribusi dalam memberikan informasi mengenai tingkat keasaman serta kandungan air dalam minyak transformator yang dapat memicu adanya elektron bebas sebagai penghantar listrik dalam isolator \cite{wahab1999novel}. Selain itu, adanya furan yang merupakan salah satu kontaminan dalam minyak transformator daya dapat memberikan informasi mengenai estimasi umur kertas isolasi.\par

Dalam setiap metode yang digunakan dalam pengujian transformator daya memiliki tujuan tertentu mengenai bagian yang ingin dilakukan pengecekan. Diagnosis kondisi keseluruhan sebuah transformator dapat dilakukan dengan menggunakan metode indeks kesehatan transformator daya yang melibatkan seluruh pengujian pada masing-masing bagian\cite{jahromi2009approach}. Dengan menggunakan metode tersebut memungkinkan dalam mengetahui kapan transformator daya harus dilakukan pemeliharaan yang berupa pergantian komponen secara akurat. Namun, penggunaan keseluruhan hasil pengujian transformator daya berdampak dalam proses diagnosis yang harus dilakukan dalam waktu yang lama. 

%Merujuk pada permasalahan di atas maka diperlukan suatu metode baru yang dapat melakuka
Peninjauan pada sisi yang lain, seiring dengan perkembangan komputer saat ini mulai dikembangkan metode komputasi dalam mengelola sebuah data. Metode tersebut dikenal dengan istilah \textit{Machine Learning} \cite{jordan2015machine}, yakni algoritma komputer yang disusun secara matematis untuk mempelajari sebuah data. Dalam perkembangannya \textit{Machine Learning} telah mengalami banyak perbaikan hingga melahirkan metode baru yang meniru sistem kerja syaraf manusia yang dikenal dengan algoritma \textit{Artificial Neural Network} (ANN) \cite{braspenning1995artificial}. Pengembangan dari ANN telah banyak disesuaikan dengan jenis data yang diolah diantaranya dalam mengolah data sekuensial adalah \textit{Recurrent Neural Network} (RNN). Data sekuensial merupakan data yang tersusun pada pola berurutan dan saling berkaitan contohnya data yang berkaitan dengan waktu. Beberapa implementasi dari RNN adalah pada prediksi beban listrik \cite{tokgoz2018rnn, tang2019application, deihimi2012application}, pengenalan suara \cite{miao2015eesen, amberkar2018speech}, atau dalam memprediksi hujan \cite{habi2019rnn}. Secara sederhana RNN tidak mampu menangani data dengan deret yang panjang karena adanya pengaruh \textit{vanishing gradient}. Modifikasi pada RNN melahirkan metode baru yakni \textit{Long Short-Term Memory} (LSTM) yang dapat mengingat semua data walaupun pada deret yang panjang sehingga dapat memprediksi secara terus menerus \cite{gers1999learning}. \par 

Merujuk pada permasalahan di atas memperlihatkan adanya sebuah solusi dalam metode diagnosis indeks kesehatan transformator daya. Dengan adanya data yang telah terkumpul pada penggunaan metode indeks kesehatan transformator daya dapat dijadikan sebuah objek data yang dapat dipelajari menggunakan \textit{Machine learning}. Hal ini memungkinkan dalam membuat sebuah sistem yang dapat menerima \textit{input} dengan beberapa pengujian yang dapat memberikan \textit{output} diagnosis indeks kesehatan transformator. Dalam perancangan sebelumnya telah digunakan metode ANN dalam kasus yang sama dengan perolehan akurasi 53.42\% untuk input DGA dan 69.86\% untuk input pengujian minyak isolasi \cite{sutaryono2015analisa}. Sehingga masih tergolong kurang optimal dalam proses diagnosis. Pada tugas akhir ini akan dirancang sebuah model LSTM yang menyesuaikan terhadap data pengujian transformator daya. Perancangan dilakukan dengan memodifikasi arsitektur LSTM sehingga diperoleh akurasi yang tinggi dalam mendiagnosis indeks kesehatan transformator daya. 

%Pendekatan dengan menggunakan \textit{machine learning} sebelumnya telah dilakukan namun dengan arsitektur berbeda seperti ANN \cite{sutaryono2015analisa, nurcahyanto2019analysis}, 

 

%yang menciptakan istilah baru yakni Deep Learning. Pada algoritma tersebut dibentuk dengan menirukan jaringan saraf pada manusia sehingga dapat mempelajari suatu data dengan baik. Perkembangan Deep Learning sendiri sudah banyak dilakukan yang menyesuaikan dengan data yang akan diproses, salah satunya adalah \textit{Long Short Term Memory} (LSTM). Algoritma ini dibentuk untuk menangani data yang bersifat sekuensial seperti halnya data yang selalu berubah seiring dengan berjalannya waktu. Keberadaan LSTM dapat menjadi solusi dalam memproses data hasil pengujian pada transformator daya sehingga hasilnya dapat memberikan klasifikasi mengenai indeks kesehatan transformator daya. Hasil klasifikasi indeks kesehatan selanjutnya dapat dimanfaatkan dalam pemeliharaan transformator daya sehingga dapat dilakukan lebih efektif.

%Data hasil pengujian transformator daya yang dikumpulkan pada waktu yang lama akan memberikan karakteristik tertentu. Hal ini tentunya memberikan informasi solutif dalam mengidentifikasikan kondisi transformator daya. Banyaknya data yang terkumpul akan sangat sulit dalam melakukan analisis dengan cara konvensional. Adanya metode komputasi modern memberikan dapat mempermudah dalam melakukan perhitungan dalam jumlah yang besar. Hal ini melahirkan suatu metode baru dalam mengenali karakteristik dari suatu data melalui perhitungan yang matematis. Metode pengenalan karakteristik data saat ini dikenal dengan Machine Learning, istilah yang diumpamakan komputer dalam mempelajari suatu data.\par

\section{Rumusan Masalah}
Merujuk pada latar belakang yang telah disampaikan, maka diperoleh rumusan masalah sebagai berikut:
\begin{enumerate}
	\item Bagaimana perancangan sistem LSTM sehingga dapat mendiagnosis indeks kesehatan transformator daya dengan akurasi yang tinggi?
	\item Bagaimana pengaruh hyperparameter dalam meningkatkan performa dari sistem yang dirancang?
\end{enumerate}

%\section{Hipotesis}
%\lipsum[4]

\section{Batasan Masalah}
Agar perancangan yang diharapkan sesuai dan dapat tercapai, maka dalam perancangan sistem tersebut, ditentukan ruang lingkup perancangan sebagai berikut:
\begin{enumerate}
	\item Proses perancangan yang diterapkan pada sistem dilakukan dalam lingkup metode Long Short Term Memory (LSTM).
	\item Hasil perancangan dapat diimplementasikan untuk diagnosis indeks kesehatan transformator daya dengan parameter \textit{input} yang dibutuhkan sesuai dengan dataset yang digunakan.
\end{enumerate}

\section{Tujuan Perancangan}
Adapun tujuan perancangan ini terdiri dari:
\begin{enumerate}
	\item Merancang sistem diagnosa indeks kesehatan pada transformator daya menggunakan metode LSTM sehingga diperoleh arsitektur yang optimal dengan akurasi yang tinggi.
	\item Menganalisis kemampuan dari hasil perancangan sistem dalam mendiagnosis indeks kesehatan transformator daya.
\end{enumerate}


\section{Manfaat Perancangan}
\begin{enumerate}
	\item Menghasilkan \textit{software} dalam membantu operator dalam mendiagnosis indeks kesehatan transformator daya
	\item Dapat melakukan penjadwalan pemeliharaan (\textit{maintenance}) pada transformator daya untuk mencegah terjadinya gangguan pada transformator daya.
	\item Dapat menjadi masukan dalam menjaga performa dari transformator daya agar dapat bekerja secara normal.
	 
\end{enumerate}

\section{Waktu Pelaksanaan Perancangan}
Pada tugas akhir ini akan dikerjakan dari proses pengajuan hingga selesai dilakukan dalam 22 minggu terhitung dari bulan januari 2021 hingga Mei 2021. Proses seluruh kegiatan disajikan pada tabel \ref{tabel:gantt chart}.
\begin{table}[h]
	
	\centering
	\caption{Waktu Pelaksanaan Tugas Akhir}
	\label{tabel:gantt chart}
	\resizebox{\linewidth}{!}{%
		\begin{tabular}{|c|c|l|l|l|l|l|l|l|l|l|l|l|l|l|l|l|l|l|l|l|l|l|l|} 
			\hline
			\multirow{2}{*}{No} & \multirow{2}{*}{Kegiatan}                                                  & \multicolumn{22}{c|}{Minggu ke-}                                                                                                                                                                                                                                                                                                                                                                                                                                                                                                                                                                                                                                                                                                                                                                                                                                                         \\ 
			\cline{3-24}
			&                                                                            & 1                                    & 2                                    & 3                                    & 4                                    & 5                                    & 6                                    & 7                                    & 8                                    & 9                                    & 10                                   & 11                                   & 12                                   & 13                                   & 14                                   & 15                                   & 16                                   & 17                                   & 18                                   & 19                                   & 20                                   & 21                                   & 22                                    \\ 
			\hline
			1                   & Penyusunan proposal TA                                                     & {\cellcolor[rgb]{0.502,0.502,0.502}} & {\cellcolor[rgb]{0.502,0.502,0.502}} &                                      &                                      &                                      &                                      &                                      &                                      &                                      &                                      &                                      &                                      &                                      &                                      &                                      &                                      &                                      &                                      &                                      &                                      &                                      &                                       \\ 
			\hline
			2                   & Pengumpulan Referensi                                                      & {\cellcolor[rgb]{0.502,0.502,0.502}} & {\cellcolor[rgb]{0.502,0.502,0.502}} &                                      &                                      &                                      &                                      &                                      &                                      &                                      &                                      &                                      &                                      &                                      &                                      &                                      &                                      &                                      &                                      &                                      &                                      &                                      &                                       \\ 
			\hline
			3                   & Pendaftaran TA                                                             &                                      & {\cellcolor[rgb]{0.502,0.502,0.502}} &                                      &                                      &                                      &                                      &                                      &                                      &                                      &                                      &                                      &                                      &                                      &                                      &                                      &                                      &                                      &                                      &                                      &                                      &                                      &                                       \\ 
			\hline
			4                   & Pengumpulan data                                                           & {\cellcolor[rgb]{0.502,0.502,0.502}} & {\cellcolor[rgb]{0.502,0.502,0.502}} & {\cellcolor[rgb]{0.502,0.502,0.502}} & {\cellcolor[rgb]{0.502,0.502,0.502}} & {\cellcolor[rgb]{0.502,0.502,0.502}} & {\cellcolor[rgb]{0.502,0.502,0.502}} & {\cellcolor[rgb]{0.502,0.502,0.502}} & {\cellcolor[rgb]{0.502,0.502,0.502}} & {\cellcolor[rgb]{0.502,0.502,0.502}} & {\cellcolor[rgb]{0.502,0.502,0.502}} &                                      &                                      &                                      &                                      &                                      &                                      &                                      &                                      &                                      &                                      &                                      &                                       \\ 
			\hline
			5                   & \begin{tabular}[c]{@{}c@{}}Penyusunan dan perancangan\\sistem\end{tabular} &                                      & {\cellcolor[rgb]{0.502,0.502,0.502}} & {\cellcolor[rgb]{0.502,0.502,0.502}} & {\cellcolor[rgb]{0.502,0.502,0.502}} & {\cellcolor[rgb]{0.502,0.502,0.502}} & {\cellcolor[rgb]{0.502,0.502,0.502}} & {\cellcolor[rgb]{0.502,0.502,0.502}} & {\cellcolor[rgb]{0.502,0.502,0.502}} & {\cellcolor[rgb]{0.502,0.502,0.502}} & {\cellcolor[rgb]{0.502,0.502,0.502}} & {\cellcolor[rgb]{0.502,0.502,0.502}} & {\cellcolor[rgb]{0.502,0.502,0.502}} & {\cellcolor[rgb]{0.502,0.502,0.502}} & {\cellcolor[rgb]{0.502,0.502,0.502}} & {\cellcolor[rgb]{0.502,0.502,0.502}} & {\cellcolor[rgb]{0.502,0.502,0.502}} & {\cellcolor[rgb]{0.502,0.502,0.502}} & {\cellcolor[rgb]{0.502,0.502,0.502}} & {\cellcolor[rgb]{0.502,0.502,0.502}} &                                      &                                      &                                       \\ 
			\hline
			6                   & Seminar Kemajuan                                                           &                                      &                                      &                                      &                                      &                                      & {\cellcolor[rgb]{0.502,0.502,0.502}} &                                      &                                      &                                      &                                      &                                      &                                      &                                      &                                      &                                      &                                      &                                      &                                      &                                      &                                      &                                      &                                       \\ 
			\hline
			7                   & Penyusunan dokumen akhir                                                   &                                      &                                      & {\cellcolor[rgb]{0.502,0.502,0.502}} & {\cellcolor[rgb]{0.502,0.502,0.502}} & {\cellcolor[rgb]{0.502,0.502,0.502}} & {\cellcolor[rgb]{0.502,0.502,0.502}} & {\cellcolor[rgb]{0.502,0.502,0.502}} & {\cellcolor[rgb]{0.502,0.502,0.502}} & {\cellcolor[rgb]{0.502,0.502,0.502}} & {\cellcolor[rgb]{0.502,0.502,0.502}} & {\cellcolor[rgb]{0.502,0.502,0.502}} & {\cellcolor[rgb]{0.502,0.502,0.502}} & {\cellcolor[rgb]{0.502,0.502,0.502}} & {\cellcolor[rgb]{0.502,0.502,0.502}} & {\cellcolor[rgb]{0.502,0.502,0.502}} & {\cellcolor[rgb]{0.502,0.502,0.502}} & {\cellcolor[rgb]{0.502,0.502,0.502}} & {\cellcolor[rgb]{0.502,0.502,0.502}} & {\cellcolor[rgb]{0.502,0.502,0.502}} & {\cellcolor[rgb]{0.502,0.502,0.502}} &                                      &                                       \\ 
			\hline
			8                   & Pendaftaran sidang TA                                                      &                                      &                                      &                                      &                                      &                                      &                                      &                                      &                                      &                                      &                                      &                                      &                                      &                                      &                                      &                                      &                                      &                                      &                                      &                                      &                                      & {\cellcolor[rgb]{0.502,0.502,0.502}} &                                       \\ 
			\hline
			9                   & Sidang TA                                                                  &                                      &                                      &                                      &                                      &                                      &                                      &                                      &                                      &                                      &                                      &                                      &                                      &                                      &                                      &                                      &                                      &                                      &                                      &                                      &                                      &                                      & {\cellcolor[rgb]{0.502,0.502,0.502}}  \\
			\hline
		\end{tabular}
	}
	
\end{table}

%\begin{ganttchart}[
%	hgrid,
%	vgrid={*{6}{draw=none}, dotted},
%	x unit=0.125cm,
%	time slot format=isodate,
%	time slot unit=day,
%	calendar week text = {W\currentweek{}},
%	bar height = 1, %necessary to make it fit the height
%	bar top shift = -0.01, %to move it inside the grid space ;)
%	]{2019-01-01}{2019-06-30}
%	\gantttitlecalendar{year, month=name, week} \\
%	\ganttbar[bar/.append style={fill=red}]{Start}{2019-01-01}{2019-01-07}\\
%	\ganttbar[bar/.append style={fill=yellow}]{A}{2019-01-08}{2019-01-14}\\
%	\ganttbar[bar/.append style={fill=cyan}]{A}{2019-01-15}{2019-01-21}\\
%	\ganttbar[bar/.append style={fill=green}]{Finish phase 1}{2019-01-22}{2019-01-28}
%\end{ganttchart}
%\lipsum[8]


\includepdf{Includes/Pemisah.pdf}
%\renewcommand{\thefootnote}{\arabic{footnote}}
\chapter{TINJAUAN PUSTAKA}
\label{BAB2:tinjauan}
\section{Transformator Daya}

Transformator daya merupakan salah satu peralatan tenaga listrik yang berfungsi dalam mentrasmisikan daya listrik dengan cara menaikkan dan menurunkan tegangan listrik untuk mengurangi rugi-rugi daya. hal ini dikarenakan rugi-rugi daya akibat impedansi yang timbul akibat jarak transmisi yang panjang dapat dikurangi dengan menaikkan tegangan. oleh karena itu dibutuhkan transformator pada sisi pembangkitan untuk menaikkan tegangan dan pada sisi penerimaan untuk menurunkan tegangan.



\begin{equation}
  L(x,W)= \frac{1}{N}\sum\limits_{i=0}^{N} l(x_i,W)   
  \label{func:loss}
\end{equation}

\section{Indeks Kesehatan Trafo}
Dalam sistem jaringan tenaga listrik pada umumnya transformator daya yang digunakan saat beroperasi yang memiliki kondisi yang baik agar terhindar dari gangguan. kondisi sebuah transformator secara keseluruhan dapat dievaluasi dengan sebuah metode yakni indeks kesehatan transformator \cite{nurcahyanto2019analysis}. metode ini merupakan hasil kombinasi data hasil inspeksi lapangan, selama beroperasi maupun hasil pengujian transformator daya di laboratorium atau lapangan \cite{ortiz2016health}. Pengujian pada transformator daya dibagi atas pengujian elektrik, pengujian kimia, dan pengujian fisik. Metode-metode yang sering digunakan pada transformator daya diantaranya adalah \textit{Dissolve Gas Analysis} (DGA), kualitas minyak transformator, furan, faktor daya, pemantauan \textit{tap changer}, riwayat pembebanan serta data pemeliharaan \cite{jahromi2009approach}.

\subsection{\textit{Dissolve Gas Analysis} (DGA)}
DGA merupakan salah satu metode yang digunakan dalam mendeteksi adanya gangguan pada transformator daya. Dalam kondisi normal dielektrik cair pada transformator daya tidak mengalami dekomposisi dengan cepat. Namun jika terjadi adanya gangguan termal atau elektrik dapat mempercepat laju dekomposisi pada dielektrik. Proses dekomposisi dapat menghasilkan gas kontaminan yang dapat mengubah sifat kondutivitas dari isolator yang dapat memicu adanya gangguan lanjutan. Secara umum terdapat beberapa jenis gas hasil dekomposisi yang dilakukan pengecekan diantaranya adalah hidrogen, 

\section{\textit{Machine Learning}}
Machine learning merupakan salah satu metode yang digunakan dalam memodelkan pola serangkaian data untuk membuat pediksi pada input yang sama atau menyerupai 
\section{\textit{Long Short Term Memory (LSTM)}}
Pada pemodelan dengan menggunakan metode RNN secara umum memiliki kemampuan dalam membuat prediksi yang dipengaruhi oleh input sebelumnya. Namun terdapat kekurangan pada metode tersebut yakni tidak mampu mengatasi dengan  seri yang panjang. misalnya pada sebuah data \textit{time series}, RNN akan sulit mengkorelasikan antara data saat ini dengan data yang sangat lampau, akibatnya jika data yang diproses dalam rentang waktu yang lama maka RNN hanya mampu membuat prediksi yang hanya berkaitan pada waktu yang pendek. kekurangan pada RNN dikarenakan adanya \textit{vanising gradient}, yakni menghapus data yang tidak berkaitan dengan data baru yang dimasukkan. Adanya kekurangan tersebut maka dibutuhkan suatu metode baru yang dapat mengingat data lampau saat menerima input terbaru. 
LSTM merupakan salah satu turunan dari pemodelan matematis yang digunakan dalam mengenali pola serangkaian data 

kelebihan yang dimiliki LSTM dibandingkan dengan RNN dikarenakan algoritma yang digunakan terdiri dari struktur yang kompleks. Secara umum terdapat 4 bagian pada arsitektur LSTM yakni \textit{forget gate}, \textit{input gate}, \textit{Cell gate}, dan \textit{Output gate}. 
\subsection{\textit{Forget Gate}}
Pada \textit{Forget gate} merupakan bagian yang menentukan mengenai informasi pada keluaran sel sebelumnya untuk dipertahankan atau dihapus. hal ini dilakukan dengan memasukkan keluaran sel sebelumnya yang digabungkan dengan masukan baru ke dalam fungsi aktivasi sigmoid. Informasi akan dipertahankan untuk hasil dari sigmoid dengan nilai 1 dan dihapus untuk keluaran yang bernilai 0. secara matematis pada \textit{forget gate} digunakan persamaan sebagai berikut:
\begin{equation}
	\boldsymbol{f_t} = \sigma_g(\boldsymbol{W_{f}}.[\boldsymbol{h_{t-1}}, \boldsymbol{x_t}] + \boldsymbol{b_f})
	\label{func:forget}
\end{equation} 
berdasarkan persamaan~(\ref{func:forget}) dapat diketahui bahwa dengan persamaan tersebut terdapat bentuk $[\boldsymbol{h_{t-1}}, \boldsymbol{x_t}]$. Hal ini merupakan operasi penggabungan vektor yakni penggabungan baris pada $\boldsymbol{h_{t-1}}$ dengan baris pada $\boldsymbol{x_t}$
\subsection{\textit{Input Gate}}
Salah satu kelebihan LSTM adalah dapat mengingat informasi data masukan yang lama. Hal ini dikarenakan karena adanya satu bagian yang berperan dalam memperbarui memori berdasarkan informasi penting dari masukan baru. kemampuan ini diperoleh karena ada dua tahapan penting pada \textit{input gate} yakni melalui lapisan \textit{sigmoid} dan \textit{tanh}. lapisan akan memberikan keluaran berupa nilai mana saja yang harus dilakukan pembaruan pada memori sedangkan lapisan \textit{tanh} memberikan keluaran berupa calon ($\boldsymbol{\tilde{C}}$) yang ditambahkan pada memori. 
\begin{equation}
	\boldsymbol{i_t} = \sigma_i(\boldsymbol{W_{f}}.[\boldsymbol{h_{t-1}}, \boldsymbol{x_t}] + \boldsymbol{b_i})\\
	\label{func:input}
\end{equation}
\begin{equation}
	\boldsymbol{\tilde{C}} = tanh(\boldsymbol{W_{C}}.[\boldsymbol{h_{t-1}}, \boldsymbol{x_t}] + \boldsymbol{b_C})
	\label{func:C-tilde}
\end{equation}
Hasil perkalian dari dua lapisan pada \textit{input gate} akan menjadi input pada memori sebagai pembaruan. pembaruan yang terjadi dalam hanya dalam jumlah yang sedikit, oleh karena itu informasi penting pada data yang lampau akan tetap tersimpan untuk jumlah data yang banyak.
\subsection{\textit{Cell gate}}
\textit{cell gate} merupakan tempat penyimpanan informasi penting pada setiap data yang diberikan pada LSTM. \textit{cell gate} terdiri dari masukan dari \textit{forget gate} untuk mengurangi informasi yang tidak diperlukan dari semua masukan sebelumnya melalui persamaan ~(\ref{func:forget}). Kemudian ditambahkan dengan hasil perkalian dari $\boldsymbol{i_t}$ dan $\boldsymbol{\tilde{C}}$.
\begin{equation}
	\boldsymbol{C_t} = \boldsymbol{f_t}*\boldsymbol{C_{t-1}} + \boldsymbol{i_t} * \boldsymbol{\tilde{C}}
	\label{func:cell_gate}
\end{equation}
Hal utama yang perlu diperhatikan adalah bahwa pada LSTM bagian \textit{cell gate} merupakan lapisan yang saling terhubung, sehingga antar sel yang berjauhan pun dapat terintegrasi. Kondisi ini yang menjadikan LSTM dapat mengatasi permasalahan versi RNN sebelumnya yang diakibatkan adanya \textit{vanishing gradient}.
\subsection{\textit{Output Gate}}
Pada bagian akhir merupakan keluaran dari sel lstm atau dapat berupa hasil prediksi berdasarkan masukan yang diberikan. Keluaran ditentukan oleh memori $\boldsymbol{C_t}$ dan masukan yang diberikan. Hal ini dilakukan dengan memasukkan  $\boldsymbol{x_t}$ dan keluaran sebelumnya ($\boldsymbol{h_{t-1}}$) pada fungsi \textit{sigmoid}. Hasil dari fungsi \textit{sigmoid} kemudian akan memfilter nilai dari \textit{cell state} yang dapat diteruskan menuju keluaran. Sebelum dikalikan dengan hasil dari gerbang \textit{sigmoid}, \textit{cell state} terlebih dahulu melewati gerbang tanh untuk mengubah nilai pada rentang -1 sampai 1. secara matematis dapat dituliskan sebagai berikut:
\begin{equation}
	\boldsymbol{o_t} = \sigma(\boldsymbol{W_i}.[\boldsymbol{h_{t-1}}, \boldsymbol{x_t}] + \boldsymbol{b_o})
	\label{func:ouput_sigmoid}
\end{equation}
\begin{equation}
	\boldsymbol{h_t} = \boldsymbol{o_t}*tanh(\boldsymbol{C_t})
	\label{func:final_output}
\end{equation}

%Informasi ini diperoleh pada lapisan sebelumnya yakni kombinasi dari \textit{forget gate} dan \textit{input gate}. Pada dasarnya 

%\begin{align}
%	f_t &= \sigma_g(W_{f} x_t + U_{f} c_{t-1} + b_f) \\
%	i_t &= \sigma_g(W_{i} x_t + U_{i} c_{t-1} + b_i) \\
%	o_t &= \sigma_g(W_{o} x_t + U_{o} c_{t-1} + b_o) \\
%	c_t &= f_t \circ c_{t-1} + i_t \circ \sigma_c(W_{c} x_t + b_c) \\
%	h_t &= o_t \circ \sigma_h(c_t)
%\end{align}
%\lipsum[5-6]
\begin{table}[h]
    \centering
    \caption{Rata-rata loss dan accuracy Model A untuk seluruh round}
\begin{tabularx}{0.95\textwidth} { 
  | >{\centering\arraybackslash}X 
  | >{\centering\arraybackslash}X | }
 \hline
  train\_accuracy &	0.46846 \\
 \hline
  train\_loss &	2.71451 \\
 \hline
  val\_accuracy &	0.47391 \\
  \hline
  val\_loss & 2.69424 \\
  \hline
\end{tabularx}
    \label{tab:my_label} 
\end{table}

\lipsum[7]
%\begin{center}
%\begin{figure}[h]
%    \includegraphics[width=\textwidth]{BAB-2/figures/alfabetbisindo.png}	
%	    \caption{Alfabet Bisindo (Almuharram, 2013).}
%	    \label{gambar:alfabet bisindo}
%\end{figure}
%\end{center}
Gambar \ref{gambar:alfabet bisindo} menyunjukkan sudut pandang umum yang digunakan dalam berkomunikasi menggunakan bahasa isyarat, yaitu tampak depan \citep{xiong2004_dscForSensorNetworks}. Sehingga, data gambar yang digunakan di penelitian ini juga memuat gestur Bisindo dari tampak depan, dengan persamaan~(\ref{func:loss}), dengan hasil penelitian di Bab~\ref{BAB4:hasil}.
\includepdf{Includes/Pemisah.pdf}
\chapter{METODE PENELITIAN}
\label{BAB3:Metode}

\lipsum[1-2]
\includepdf{Includes/Pemisah.pdf}
\chapter{HASIL DAN PEMBAHASAN}
\label{BAB4:hasil}

\lipsum[3-4]

\includepdf{Includes/Pemisah.pdf}
\chapter{KESIMPULAN DAN SARAN}

\section{Kesimpulan}
\lipsum[6]

\section{Saran}
\lipsum[9]
\includepdf{Includes/Pemisah.pdf}

%------------------------------------------------------------
% BIBLIOGRAPHY %
%------------------------------------------------------------
\begin{singlespace}
  \bibliographystyle{IEEEtranN}
  \addcontentsline{toc}{chapter}{DAFTAR PUSTAKA}
  %\addcontentsline{toc}{chapter}{Index}
  \bibliography{Referensi}
\end{singlespace}

\includepdf{Includes/Pemisah.pdf}

%% Appendices
\appendix
%\Appendix{Empirical Binary Entropies for Given Bit-Flipping Probabilities}
\label{AppendixA:binaryEntropies}

With simulation setup given in Figure~\ref{fig:p_abc}, the empirical binary entropies for given bit-flipping probabilities are shown in Figure~2 \citep{xiong2004_dscForSensorNetworks}.
\begin{figure}[h]
\centering \includegraphics[scale=1]{Lampiran-A/figures/binaryEntropiesSystemSetup-eps-converted-to.pdf}
  \caption{System setup for obtaining $p_{ABC}$.}
  \label{fig:p_abc}	
\end{figure}
% tambahkan sendiri folder-folder (lampiran) yang lain

\backmatter

\end{document}
