\chapter{KESIMPULAN DAN SARAN}

\section{Kesimpulan}
Berdasarkan perolehan hasil pada setiap percobaan serta analisis yang telah dipaparkan pada bab-bab sebelumnya, maka pada perancangan ini dapat ditarik kesimpulan sebagai berikut:
\begin{enumerate}
	\item Pada penggunaan set data yang bersumber dari \cite{shah2016predict} dalam membentuk sistem diagnosis indeks kesehatan transformator daya denngan menggunakan LSTM diperoleh arsitektur terbaik dengan menggunakan 2 \textit{hidden layer} dan penggunaan fungsi aktivasi berupa \textit{relu}. Arsitektur model LSTM tersebut mampu memberikan hasil akurasi pengujian 99\% pada penggunaan rasio set data pelatihan:pengujian adalah 7:3.
	\item Kasus kedua pada perancangan sistem diagnosis indeks kesehatan transformator daya dengan menggunakan set data \cite{abu2012calculation}, model LSTM terbaik berdasarkan hasil percobaan diperoleh ketika digunakan \textit{hidden layer} tunggal dan menggunakan fungsi aktivasi \textit{selu}. Penggunaan konfigurasi \textit{hyperparameter} tersebut mampu menghasilkan model dengan yang dapat mendiagnosis indeks kesehatan transformator daya tanpa kesalahan berdasarkan \textit{input} data pengujian.
	\item Berdasarkan hasil percobaan pada kedua kasus yang telah dilakukan maka dapat diperoleh informasi bahwa baik pada kasus 1 dan 2 penggunaan fungsi aktivasi \textit{relu} merupakan yang terbaik pada kedua sistem. kategori dengan jumlah data yang sedikit (tidak berimbang) lebih sulit didiagnosis oleh sistem dibandingkan data kategori yang lebih banyak. Penggunaan rasio pelatihan:pengujian sebesar 7:3 lebih cocok digunakan untuk jumlah set data yang sedikit, sedangkan untuk data yang besar lebih bagus untuk menggunakan rasio 8:2.
\end{enumerate}

\section{Saran}
Adapun saran yang dapat diberikan dalam pengembangan diagnosis indeks kesehatan transformator daya adalah sebagai berikut:
\begin{enumerate}
	\item Set data yang digunakan dalam percobaan sebaiknya merupakan data yang seimbang jumlah data targetnya.
	\item Menggunakan metode penyeimbangan data jika data yang digunakan merupakan data tidak berimbang sebelum dilakukan proses pelatihan.
	\item Melakukan optimasi pada bobot (\textit{weight}) serta bias agar hasil akurasi setiap percobaan selalu sama, sehingga dapat mengurangi jumlah percobaan dengan arsitektur yang sama.
\end{enumerate}