Transformator daya merupakan salah satu komponen yang fundamental dalam sistem transmisi jaringan listrik. Hal ini karena pada dasarnya transformator daya berperan dalam mengurangi rugi-rugi daya pada proses transmisi yakni dengan menaikkan tegangan selama proses transmisi pada jarak yang sangat jauh. Oleh karena itu kegiatan pemeliharaan dalam memastikan kondisi transformator daya dapat bekerja dengan baik menjadi penting dilakukan. Namun, proses menentukan kondisi transformator berlangsung dalam waktu yang panjang serta harus dilakukan oleh teknisi yang handal. Pada Tugas Akhir ini dilakukan sebuah perancangan diagnosis indeks kesehatan transformator daya berbasis \textit{machine learning} dengan metode \textit{Long Short-Term Memory} (LSTM). Perancangan dilakukan dengan beberapa perubahan pada jumlah \textit{hidden layer}, fungsi aktivasi serta perubahan rasio set data yang digunakan selama pelatihan dan pengujian. Hasilnya, pada kasus pertama diperoleh akurasi terbaik 99\% pada proses pelatihan dan pada kasus kedua diperoleh model yang mendiagosis tanpa kesalahan saat pengujian. Model terbaik yang dihasilkan selanjutnya diimplementasikan pada sebuah aplikasi yang ditanamkan dalam perangkat digital.

Kata kunci: Indeks Kesehatan, Transformator Daya, \textit{Machine learning}, LSTM.