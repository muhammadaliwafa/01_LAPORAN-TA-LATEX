Puji dan syukur penulis panjatkan kehadirat Allah SWT, karena berkat rahmat dan karunia-Nya penulis dapat menyelesaikan laporan Tugas Akhir dengan judul Diagnosis Indeks Kesehatan Transformator Daya Menggunakan  \textit{Long Short Term Memory} (LSTM). Keberhasilan dalam penyusunan laporan ini tentunya tidak akan terwujud dan terselesaikan dengan sangat baik tanpa adanya bimbingan, bantuan dan dorongan dari berbagai pihak, baik itu secara material maupun spiritual. \par
Dengan segala ketulusan dan kerendahan hati, penulis ingin menyampaikan banyak terima kasih kepada semua pihak yang terlibat dalam penulisan laporan Tugas Akhir ini. Oleh karena itu penulis sampaikan ucapak terima kasih kepada:

\begin{enumerate}
	\item Orang tua serta keluarga yang senantiasa mendoakan serta memberikan dukungan dan motivasi selama proses pengerjaan dan penyelesaian Tugas Akhir.
	\item Bapak Dr. Eng. Muhammad Abdillah, S.T, M.T. selaku pembimbing I dan Bapak Teguh Aryo
	Nugroho, M.T selaku dosen pembimbing II yang bersedia meluangkan banyak waktu dalam memberikan bimbingan, arahan serta motivasi kepada penulis.
	\item  Dr.Eng. Wahyu Kunto Wibowo S.T., M.Eng selaku Ketua Program Studi Teknik Elektro, Universitas Pertamina serta sebagai dosen wali yang telah banyak memberikan motivasi dan dorongan selama masa perkuliahan.
	\item Ibu Aulia Rahma Annisa, S.T., M.T., selaku Koordinator Kemahasiswaan Program Studi Program Studi Teknik Elektro Universitas Pertamina.
	\item Seluruh jajaran dosen dan staf Program Studi Teknik Elektro Universitas
	Pertamina.
	\item Seluruh pihak yang tidak dapat penulis sebutkan yang terlibat secara langsung maupun tidak langsung.
\end{enumerate}
Penulis menyadari bahwa dalam penyusunan laporan Tugas Akhir ini masih banyak terdapat kekurangan yang disebabkan oleh banyak faktor dan keterbatasan penulis. Akhir kata, terimakasih dan semoga hasil dari Tugas Akhir ini dapat bermanfaat bagi semua pihak.

	\vskip 1cm
	\hspace{10cm} Jakarta, 27 Mei 2021 \par
	\vskip 3cm
	\hspace{10cm} Muhammad Ali Wafa